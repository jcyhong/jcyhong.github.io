% LaTeX file for resume 
% This file uses the resume document class (res.cls)

\documentclass{res} 
%\usepackage{helvetica} % uses helvetica postscript font (download helvetica.sty)
%\usepackage{newcent}   % uses new century schoolbook postscript font
\usepackage{multicol} 
\setlength{\textheight}{9.5in} % increase text height to fit on 1-page 
\usepackage{geometry}
 \geometry{
 a4paper,
 total={170mm,257mm},
 left=15mm,
 top=15mm,
 right = 15mm
 }

\begin{document} 
\newcommand*\leftright[2]{%
  \leavevmode
  \rlap{#1}%
  \hspace{0.6\linewidth}%
  #2}
\newcommand*\leftrighta[2]{%
  \leavevmode
  \rlap{#1}%
  \hspace{0.5\linewidth}%
  #2}

\name{Chun Yu Hong (Johnny)\\[12pt]}    % the \\[12pt] adds a blank
				        % line after name      

\address{393 Evans Hall, Berkeley, CA 94720\\Phone: (510) 676-8616\\Email: jcyhong@berkeley.edu\\Website: jcyhong.github.io}
                                  
\begin{resume}  
 
\section{EDUCATION}
    \leftright{\textbf{Ph.D. student in Statistics}}{August 2014 - Present} \\ 
    \leftright{University of California, Berkeley}{{\it Expected May 2020}} \\
    Research interests: high-dimensional covariance matrix estimation; latent variable models\\
    Advisors: William Fithian (Department of Statistics), Perry de Valpine (Department of Environmental \\Science, Policy, and Management)
%    \\
%    Potential relevant coursework: Statistical Models: Theory and Application (STAT 215A), \\Statistical Learning Theory (STAT 241A)
    \\
    \\
    \leftright{\textbf{B.S. in Applied Mathematics (with Honors)}}{September 2011 - June 2014} \\
    \textbf{B.S. in Statistics (with Honors)} \\
    University of California, Davis \\
%    Potential relevant coursework: Statistical Computing (R) (STA 141), Regression Analysis (STA 108), \\ 
%    Applied Time Series Analysis (STA 137), Bayesian Statistical Inference (STA 145)
\section{EXPERIENCE}

    \leftright{{\bf Data Science Analytics Intern}}{May 2019 - August 2019} \\ 
    \leftright{Facebook}{{Menlo Park, CA}} 
            \begin{itemize}\setlength\itemsep{0em}
        \item[-] Worked closely with cross-functional partners on analyzing the performance of \\video poll ads. 
\item[-]  Developed a novel framework for quantifying value of users to different types of advertising. 
\item[-] Explored the value of brand advertising on Facebook and identified areas in which \\brand ads performed particularly well. 
        \end{itemize}

    \leftright{{\bf Research Scientist}}{September 2018 - December 2018} \\ 
    \leftright{PRO Unlimited @ Facebook}{{Menlo Park, CA}} 
            \begin{itemize}\setlength\itemsep{0em}
        \item[-] Evaluated the extent of treatment effect heterogeneity in Facebook experiments and \\raised awareness of the issue of multiple testing.
        \end{itemize}
  

    \leftright{{\bf Graduate Student Instructor}}{September 2014 - December 2018} \\ 
    \leftright{UC Berkeley Department of Statistics}{{Berkeley, CA}} 
            \begin{itemize}\setlength\itemsep{0em}
        \item[-] Grades exams, and occasionally suggests exam questions. 
        \item[-] Holds weekly lab sections and office hours, answering students' questions about the class materials.
        \item[-] Courses: STAT 133 (Concepts in Computing with Data), STAT 134 (Concepts of Probability), 
        \item[] STAT 135 (Concepts of Statistics), 
        STAT 153 (Introduction to Time Series Analysis), 
        \item[] STAT 154 (Modern Statistical Prediction and Machine Learning), 
        \item[] STAT 210A (Theoretical Statistics) (Grader), 
         STAT 222 (Statistics MA Capstone Project)
        \end{itemize}
  
    \leftright{{\bf Data Scientist Intern}}{May 2018 - August 2018} \\
    \leftright{Quora Inc.}{{Mountain View, CA}}
    \begin{itemize}
     \item[-] Improved accuracy and precision in the A/B testing framework via advanced statistical methods.
     \item[-] Conducted experiment analyses and metric investigations.
     \item[-] Developed a data-driven approach for Quora digest email frequencies.
     \end{itemize}  
   
    \leftright{{\bf Data Science Intern}}{June - August 2016; June - August 2017} \\
    \leftright{Adobe Systems Incorporated}{{San Jose, CA}}
    \begin{itemize}
     \item[-] Developed models for customer churn forecasting using time series analysis and machine learning.
     \item[-] Built an R package for finding the optimal combination of multiple forecasts.
     \item[-] Created interactive visualization of model performance via R shinyApp.
     \item[-] Conducted performance evaluation of the existing marketing lead scoring system.
     \end{itemize}
     
    \leftright{{\bf Statistical Consultant}}{January 2016 - May 2016} \\ 
    \leftright{UC Berkeley Department of Statistics}{{Berkeley, CA}}
    \begin{itemize}
     \item[-] Provided statistical guidance for researchers (primarily for UC Berkeley students) in various \\disciplines, such as psychology, biology, and economics.
     \item[-] Discussed statistical issues such as experimental design and hypothesis testing procedures.    
    \end{itemize}        
    
    \leftright{{\bf Undergraduate Researcher}}{August 2013 - September 2013} \\ 
    \leftright{UC Davis Department of Mathematics}{{Davis, CA}} 
        \begin{itemize}
        \item[-] Developed the first version of the program in Sage for computation and experimentation
       \\with the 1-row Gomory-Johnson infinite group problem.
       \item[-] Advisor: Matthias K\"{o}ppe.
        \end{itemize}
%    
%    \leftright{{\bf Mathematics Tutor}}{November 2012 - August 2014} \\ 
%    \leftright{Student Academic Success Center}{{Davis, CA}}
%    \begin{itemize}
%     \item[-] Emphasized interactive learning by asking students questions instead of routinely presenting\\ 
%     the solutions when solving problems.        
%    \end{itemize}

\section{SELECTED WORKS}
\textbf{Relaxed Wasserstein with Applications to GANs}, Xin Guo, Johnny Hong, Tianyi Lin, and Nan \\Yang, 2017. Preprint.
\\
\textbf{Sampling-Based Approaches to Maximum Likelihood Estimation for Latent Variable \\Models}, Johnny Hong, Sara Stoudt, and Perry de Valpine, 2017. Under Revision.

\section{PROJECTS}
\textbf{An introduction to the use of hidden Markov models for stock return analysis}, \\Johnny Hong and Yannik Pitcan, 2015.
            \begin{itemize}\setlength\itemsep{0em}
            \item[-] Final group project for the graduate-level course Statistical Learning Theory
            \item[-] Project Role: Developed a hidden Markov model (HMM) for volatility analysis of stock returns
            \end{itemize}
            
\section{COMPUTER SKILLS}          
    Proficient (had used extensively at work before) in R, Python (including packages such as matplotlib, \\numpy, and pandas), SQL, Amazon Redshift, PySpark, Presto. \\
    Experience (mainly from undergraduate coursework) in C, C++, and MATLAB. \\

\section{EXAMS}
    \leftright{Actuarial Exam P (Probability): Pass (Grade: 10)}{July 2013} 
 
\section{HONORS AND AWARDS}        
    \leftright{Outstanding Graduate Student Instructor Award}{UC Berkeley; 2016 - 2017}\\
    \leftright{Dean's List}{UC Davis; Fall 2011 - June 2014} \\
    \leftright{Joseph Bonnheim Memorial Scholarship}{UC Davis; Spring 2012, Spring 2013} \\
    \leftright{Eric C. Ruliffson Scholarship in Mathematics}{UC Davis; Spring 2012, Spring 2013}\\
     \leftright{James and Leta Fulmor Scholarship}{UC Davis; Spring 2012}\\
     \leftright{Robert Lewis Wasser Memorial Scholarship}{UC Davis; Spring 2012}      
 
\section{VOLUNTARY EXPERIENCE}   
 \leftright{{\bf Climate Datathon Mentor}}{April 2019} \\ 
    \leftright{Berkeley-Haas AMENA Center}{{Berkeley, CA}}
    \begin{itemize}\setlength\itemsep{0em}
    \item[-] Helped participating teams brainstorm data analysis ideas.
    \item[-] Addressed technical concerns arise in the competition.
    \end{itemize}

 \leftright{{\bf DataFest Helper}}{April 2017, April 2018} \\ 
    \leftright{University of California, Berkeley}{{Berkeley, CA}}
    \begin{itemize}\setlength\itemsep{0em}
    \item[-] Helped coordinate a data analysis competition for undergraduates.
    \item[-] Provided suggestions and feedback to participants.
    \end{itemize}
%    \leftright{{\bf Math Circle Teaching Assistant}}{January 2013 - March 2013} \\ 
%    \leftright{University of California, Davis}{{Davis, CA}}
%    \begin{itemize}\setlength\itemsep{0em}
%    \item[-]Worked with a graduate student in teaching high school students elementary graph theory.
%    \item[-]Revised lesson plans and worksheets authored by the graduate student.
%    \end{itemize}
    
\section{SELECTED PRESENTATIONS}
 \leftright{{\bf International Statistical Ecology Conference (ISEC)}}{July 2018} \\ 
    \leftright{University of St Andrews}{{St Andrews, Fife, Scotland}}
    \begin{itemize}\setlength\itemsep{0em}
    \item[-] Poster presentation of \textbf{Sampling-Based Approaches to Maximum Likelihood \\Estimation for Latent Variable Models}, joint work with Sara Stoudt and Perry de \\Valpine.
    \end{itemize}
    
% \leftright{{\bf Berkeley Statistics Annual Research Symposium (BSTARS)}}{March 2018} \\ 
%    \leftright{University of California, Berkeley}{{Berkeley, CA}}
%    \begin{itemize}\setlength\itemsep{0em}
%    \item[-] Thunder talk and poster presentation of \textbf{A Spectral Approach to Incorporate \\Phylogenetic Signals}, joint work with Eoin Brodie, Perry de Valpine, William Fithian,\\ and Ulas Karaoz.
%    \end{itemize}
%   
% \leftright{{\bf Berkeley Statistics Annual Research Symposium (BSTARS)}}{March 2017} \\ 
%    \leftright{University of California, Berkeley}{{Berkeley, CA}}
%    \begin{itemize}\setlength\itemsep{0em}
%    \item[-] Poster presentation of \textbf{Sampling-Based Approaches to Maximum Likelihood \\Estimation for Latent Variable Models}, joint work with Sara Stoudt and Perry de \\Valpine.
%    \end{itemize}
%    
\end{resume}
\end{document}
